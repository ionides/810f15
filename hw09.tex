\documentclass[12pt]{article}
\usepackage{fullpage,url}\setlength{\parskip}{3mm}\setlength{\parindent}{0mm}
\begin{document}

\begin{center}\bf
Homework 9. Due by 5pm on Wednesday 11/4.

Computing background on R, C and Unix/Linux, and techniques for effective and responsible research involving computation.

\end{center}

Here are some questions on what you know and what you might be interested in learning about, with regards to the most widely used scientific computing environments in the field of Statistics. Please write brief answers to the following questions, by editing the tex file at \url{dept.stat.lsa.umich.edu/~ionides/810/hw09.tex}, and send me the resulting pdf file. 

\begin{enumerate}

\item Take a look through the chapter headings for the R manual, at
\url{http://cran.r-project.org/doc/manuals/R-intro.html}. Which chapters look fairly familiar to you? Are there specific chapters or topics that you think, from your perspective, it would be particularly useful (or particuarly worthless) to discuss in class?

YOUR ANSWER HERE

\item Some familiarity with command line Unix/Linux is a basic research skill. Take a look over the tutorials at \url{http://www.ee.surrey.ac.uk/Teaching/Unix}. Does the material in Tutorials 1--5 seem routine, somewhat familiar, or entirely unfamiliar?

YOUR ANSWER HERE

\item Often it is convenient to link some simple C code with R to make the most computationally intensive part of an analysis run quickly. How much programming have you done in a low-level language like C or C$++$? 

YOUR ANSWER HERE

\item Does the sample C code on the course website (\url{http://dept.stat.lsa.umich.edu/~ionides/810/gompertz.c}) seem readable to you? Are you comfortable with the use of pointers (passing by reference versus passing by value) in this code?

YOUR ANSWER HERE

(This code uses the \texttt{Rmath.h} header file to access versions of various R functions from within C. This is described at \url{http://cran.r-project.org/doc/manuals/R-exts.html})


\item Later, we will also discuss version control (specifically, git), internet code development repositories (specifically, github) and reproducible computing (specifically, knitr).  Two relevant links, which you do not need to read carefully at this point, are a tutorial by Karl Broman,

\url{http://kbroman.org/github_tutorial} 

and the first part of Matthew Stephens' 2014 IMS Medallion lecture,

\url{https://github.com/stephens999/ash/blob/master/talks/ASC2014.pdf?raw=true}

Please let me know if you have any experience with these concepts and/or these specific implementations. I'm anticipating that they will be new to most.

YOUR ANSWER HERE

\end{enumerate}
\end{document}
