\documentclass[12pt]{article}
\usepackage{fullpage,url}\setlength{\parskip}{3mm}\setlength{\parindent}{0mm}
\begin{document}

\begin{center}\bf
Name: YOUR NAME HERE

Homework 2. Due by 5pm on Wednesday 9/16.

Building and maintaining healthy mentor/mentee relationships

\end{center}

Read pages 4--7 of {\em On Being a Scientist}. Write brief answers to the following questions, by editing the tex file at \url{dept.stat.lsa.umich.edu/~ionides/810/hw02.tex}, and send me the resulting pdf file. We are primarily interested in the mentorship relationship between a PhD student and their thesis adviser, but you may also consider the broader academic context.

\begin{enumerate}
\item What roles do mentorship relationships play in professional development of PhD students? 

YOUR ANSWER HERE

\item What do the mentee and mentor gain from the relationship?

YOUR ANSWER HERE

\item Describe a situation in which the interests of the mentor and mentee are aligned. [In this context, ``interests'' means career or financial advancement.]

YOUR ANSWER HERE

\item Describe a situation in which the interests of the mentor and mentee are conflicting. 

YOUR ANSWER HERE

\item How are mentorship relationships initiated? E.g., how do you find a thesis adviser?

YOUR ANSWER HERE

\item Collaboration: What are the advantages and disadvantages for an aspiring Statistician of building a mentorship relationship with a researcher who is not a Statistician? This could be in addition to, or instead of, having a mentor who is a Statistician.

YOUR ANSWER HERE

\item Describe a way in which a mentorship relationship can turn unhealthy. What warning signs should one look for? What actions can one take?

YOUR ANSWER HERE

\end{enumerate}

\end{document}
