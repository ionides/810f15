%documentclass[11pt]{seminar}
\documentclass[portrait,11pt]{seminar}
\usepackage{url}
\slidefontsizes{10}

\newcommand\bs{\begin{slide*}}
\newcommand\es{\end{slide*}}

\newcommand\bi{\begin{myitemize}}
\newcommand\ei{\end{myitemize}}


\usepackage{ulem}
\usepackage{amsmath,amssymb,amsfonts,amsthm,graphicx}

\usepackage{color,semcolor}
\definecolor{green}{rgb}{0,0.8,0.2}

\newcommand\prob{\mathbb{P}}
\newcommand\E{\mathbb{E}}
\newcommand\SampleSpace{\mathbb{S}}
\newcommand\R{\mathbb{R}}
\newcommand\Z{\mathbb{Z}}
\newcommand\Var{\mathrm{Var}}
\newcommand\Cov{\mathrm{Cov}}
%\newcommand\mydefinition[1]{{\ \uwave{#1}}}
%\newcommand\mydefinition[1]{{\red \textbf{#1}}}
\newcommand\mydefinition[1]{{\textbf{#1}}}
\newcommand\mymath{\blue }
%\newcommand\myproof{\underline{Proof:} }
\newcommand\myproof{{Proof:} }
\newcommand\equals{{=}\,}
\newcommand\given{{\, | \,}}

\newcommand\hd[1]{\centerline{\large\bf #1}}
\newcommand\shd[1]{\underline{\large #1}}

\slideframe{none}
\newenvironment {myitemize} {
                 \begin{list}{$\bullet$ \hfill}
                 {\setlength{\labelwidth}{0.3 cm}
                  %\setlength{\leftmargin}{0em}
                  \setlength{\leftmargin}{0.15cm}
                  \setlength{\itemindent}{0.15cm}
                  \setlength{\labelsep}{0cm}
                  \setlength{\parsep}{0.2 ex}
%                  \setlength{\itemsep}{0.25 cm}
                  \setlength{\itemsep}{0.15 cm}
      \setlength{\topsep}{0.1cm}}} %space between title and 1st item
   {\end{list}}

\newenvironment {myequation} {\vspace{-1mm}\begin{equation*}}{\end{equation*}\vspace{1mm}}

\newenvironment {myeqnarray} {\vspace{-1mm}\mymath \begin{eqnarray*}}{\end{eqnarray*}\vspace{-1mm}}




\begin{document}
\bs

{\bf
Publication and peer review

Slides are posted at \url{http://dept.stat.lsa.umich.edu/~ionides/810/}
}

\medskip

{\bf  What roles does peer reviewed publication play in the current infrastructure of science? }

{\it ``In modern science, the peer reviewed publication is the key and fundamental element. Almost all research results are built on it, which means that they are desseminated, improved and recognized through the peer reviewed publication at most cases.''}


% 2014 \it  ``Currently, peer reviewed publications are the most widely accepted way to submit papers. By publishing a peer reviewed paper, other scientists are now able to both use what you have developed and trust that your research is credible.''

% 2013 1. ``A peer revision plays a very important in the system, since to be able to continue improving science, people should rely on previous work, and peer revision plays the role of assure other scientist that research is assured.''

% 2013 2. ``Such processes can ensure the quality and validity of publication to some extent. This could maintain an stable and relative high error bar for most scientific findings. So this is definitely an important guardrail for modern scientific research. However, occasionally, some really significant discoveries might also be hidden by this as such things might be not easy to accepted by others immediately or out of the scope of well-established criteria. For instance, it is said that the paper of Benjamini and Hochberg for FDR control was originally rejected by many peer-review based journals in the first a few years.''

% 2013 3. ``Peer review is a way for the scientic community to appraise studies. This supposedly upholds a high standard for scientic work.''



% 2013 3. ``Peer review offers a valuable way of evaluating and improving the quality of scientific papers. If research results are made available to other researchers or to the public before publication in a journal, researchers need to use some kind of peer review process that may compensate for the lack of the formal journal process.''



% 2013 4. Peer reviewed publication makes show new discoveries to others while assuring their authors credit possible in the current science world. It is the most important way of disseminating a complete set of research results and useful tools to give proper citation. Also, it offers a valuable way of evaluating and improving the quality of scientific papers.

\es
\bs
{\bf  What roles does peer reviewed publication play in the current infrastructure of science? }

{\it ``The peer review process is crucial to quality control in the literature. So crucial, in fact, that publications that are not peer-reviewed are treated with heavy skepticism.''}

\medskip

{\it ``The peer-reviewed rule is a standard that only admits the peer-reviewed publication; any new discovery that is not written out and submitted is not admitted, which also avoids other claiming priority.''}

%\medskip

\es
\bs

SCIENCE WORKS INCREMENTALLY. BY CONVENTION, WE ARE ALLOWED TO BUILD ON PEER-REVIEWED RESULTS IN OUR OWN WORK, ASSUMING THAT THEY ARE CORRECT. IF OUR RESULTS DEPEND ON THE CORRECTNESS OF PREVIOUS NON-PEER-REVIEWED RESULTS, THERE IS NO SUCH PRESUMPTION OF CORRECTNESS.

\medskip

IN PRACTICE THIS MUST BE BALANCED WITH UNDERSTANDING THAT MANY PEER-REVIEWED PAPERS CONTAIN INCORRECT RESULTS.
A RECENT  INVESTIGATION ({\bf Science, 28 August 2015, ``Estimating the reproducibility of psychological science''}) FOUND THAT ONLY 40\% OF RESULTS PUBLISHED IN TOP JOURNALS COULD BE REPRODUCED SUCCESSFULLY.

\medskip

HOW DO YOU THINK THIS COMPARES WITH THE STATISTICS LITERATURE?

\es 
\bs {\bf How does one choose a reasonable balance in research between (i) quality versus quantity; (ii) timeliness versus thoroughness?}

{\it ``Both quality and number of publications are important for career of a researcher. It is good to publish some intermediate publications for an on-going project. This helps researcher to improve the quality of his/her research by receiving feedback from his/her colleagues. On the other hand, if researcher just care about the number of publication, he will publish tons of publications that they won't be read or cited. If researcher has obsession about the quality of research, he will hardly have any publication.''}

% Each publication should improve the scientic knowledge while also not attempting to answer every question on a specic subject, which can help balance the various duties of a researcher.
 
% 2014 \it ``(i) quality versus quantity: It might depends on what one is pursuing and what the situation is. For example, if one wants to devote into academia, some high quality papers is more important in demonstrating one's potential and ability, than considerable numbers of mediocre or poor papers.

% 2014 (ii) timeliness versus thoroughness. To balance between these two, one has to consider whether the current work progress is good enough to make a impact on the current academia. If it does, then one might just stop and get the results published, in order to promote the science.''

\es
\bs
How does one choose a reasonable balance in research between (i) quality versus quantity; (ii) timeliness versus thoroughness?

{\it ``It is impossible to be certain that a manuscript or piece of code or a dataset or any scientific product is perfect. Perfection is not a reasonable goal. The goal should be to minimize errors and fully document each step along the research project so that if a question arises later it can be investigated and addressed. In general, I think we should err toward valuing quality more than quantity and thoroughness more than timeliness, because science is a slow process and problems happen when people rush.''}


% 2013 1. ``There is a possibility that thoroughness takes too much time. Therefore, on the basis of having substantial research results, which might be imperfect or incomplete at the current stage but still worth being published, it might be a good idea to get it published and researchers can continue to work on this project to perfect the results.''

% 2013 2. ``publishing unfinished results should be always avoided''

\es
\bs
{\it
``Quality vs. Quantity: the researchers at their beginning stages could pursue the
quantity of papers. The beginning needs to be familiar with the whole process of doing
research work as soon as possible. They needs practice to do research work. Besides,
the beginner usually does not have a good understanding of the research currents, it
is relatively hard for them to publish high-quality papers. For the senior researchers,
they usually have gotten a large quantity of papers. They should pay more attention
to the quality of their new papers.

Timeliness vs. Thoroughness: Do the research step by step. In the beginning, the
researcher could do some simple works, for example publish some easy papers. In
this way, they could have a more concrete or better and better understanding of their
research programs. Besides, publishing a few easy papers provides them with a good
opportunity to exchange ideas with other peer researchers, especially when they attend
a conference or workshop etc.''
}
\es
\bs
{\it ``The bottom line is that the researcher should not publish as long as he/she has any doubt on his/her statement. At the same time, however, it is also necessary to open the results before accomplishing perfect thoroughness because other researchers can provide new ideas to resolve the problem.''}


\es
\bs
{\bf How does one choose a reasonable balance in research between (i) quality versus quantity; (ii) timeliness versus thoroughness?}

{\it ``Quality and thoroughness are always crucial when working on research. If the research is of low quality, it will not be respected as much and your reputation could be harmed. At the same time, you need to make sure that you aren’t so concerned with the thoroughness and quality that the research either becomes outdated or you have missed other opportunities to conduct important research while fixing minor details.''}


\es \bs To what extent are referees responsible for checking the correctness of research? 

% 2014 {\it  ``I don't think this is mentioned explicitly in the book. But to my understanding, I think the referees should do their best to make sure the paper is correct, because their reputation is at stake.'' }

{\bf A}. {\it ``In theoretical papers, referees are responsible for checking the correctness of proofs. In applied work, the referee should be skeptical and critical of design and analysis of experiments. However, they should not be required to go line-by-line through code in an attempt to discover bugs and/or inconsistencies with the text.''}

\medskip

{\bf B}. {\it ``It is mainly up to the authors of paper to make sure the correctness of the research, but the referees also take responsibility to check the correctness of the research, and they should avoid the incorrect or inaccurate results being published.''}

\es
\bs  To what extent are referees responsible for checking the correctness of research? 


{\it ``Ideally, the referee’s job is to help produce quality science. The reviewer should attempt to be helpful and reasonably thorough in writing a response to the manuscript, but cannot be expected to do the work herself. Even if the review is negative, the referee should attempt to provide helpful suggestions or a reasonable description of why she voted to reject the paper. The time taken on a review should be enough to provide constructive criticism on the work as a whole.''}


% 2014 Most people thought referees are responsible for checking correctness. Here is one dissent:


% 2014 {\it  ``Ultimately the authors are responsible for the correctness of the research. Peer review can catch glaring problems and shape the paper to have greater impact, but time and journal space restrictions prevent the detection of all mistakes.''

% 2013 `` Referees are supposed to be responsible, but the brunt of the responsibility falls on the authors. One shouldn't expect others to be extremely meticulous in treating another person's work, in most cases.''

% 2013. ``Referees should guarentee there is no mistake in the research he/she reviews. More specifficly, he/she should try their best to achieve this goal and whenever things go out of his/her knowledge, he should consult correct person.''



\es \bs {\bf How should a responsible referee decide how much time to take writing a review? }

{\bf A}. {\it  ``A responsible referee should spend enough time on writing a review so that he or she can understand the article thoroughly, can find any error in the paper, and can give suggestions for improvement.''}

\medskip

{\bf B}. {\it ``Ideally, the referee’s job is to help produce quality science. The reviewer should attempt to be helpful and reasonably thorough in writing a response to the manuscript, but cannot be expected to do the work herself. Even if the review is negative, the referee should attempt to provide helpful suggestions or a reasonable description of why she voted to reject the paper. The time taken on a review should be enough to provide constructive criticism on the work as a whole.''}

\medskip

HOW WOULD YOU DESCRIBE THE DIFFERENCES BETWEEN THESE TWO RESPONSES. WHICH DO YOU AGREE WITH MORE? OR NEITHER? OR BOTH?

\es
\bs

{\it  ``The content of each paper might dictate how long is spent writing a review. A longer paper with a more complicated data analysis will require more reviewing time. Papers that make especially surprising or strong claims might also require more time for review. This is not to imply that the time spent reviewing a paper will be proportional to its length or the renown of its authors. Judgment is required in every case.''}

% 2014 \it ``A referee should take enough time for him/her to evaluate all the content of the paper thoroughly and to submit a meaningful review.''

% 2014 ``It should be depend on the complexity or size of the work.''


% 2013 1. ``Take as much time as necessary.''



% 2013 2. ``A responsible referee should give constructive opinions about the paper. Therefore, the referee should share a decent amount of time on writing a review, but based on the fact that his/her own schedule is not greatly interrupted.''


\es \bs {\bf What are the costs and benefits of agreeing to review a paper?}

{\it ``The benefits of reviewing a paper is much more than its costs. Reviewing process provides a communication channel between reviewer and researcher which help both of them to get ideas for their future research. One cost to reviewing a paper is the time that a reviewer spends to read and evaluate the quality of a paper.''}

\medskip EMPIRICALLY, QUITE A LARGE FRACTION OF PEOPLE DECLINE INVITATIONS TO REFEREE A PAPER. GENERALLY, FULL PROFESSORS ARE UNLIKELY TO AGREE TO REFEREE; PHD STUDENTS AND POSTDOCS ARE LIKELY TO AGREE; THERE IS SOME INTERPOLATION BETWEEN THESE ENDPOINTS. WHY IS THIS?

\es
\bs
{\bf What are the costs and benefits of agreeing to review a paper?}

{\it ``The costs are time spent on work not necessarily related to one’s research interests. However, a good referee is well recognized and respected within the community. Doing a good job as a referee may mean promotion to more presitigious editorial positions within top ranked journals.''}


{\it ``Reviewing papers often seems that it only has costs: it takes significant time and effort, reviews are generally anonymous so there are limited reputational benefits, and there are more efficient ways to stay current with the literature. But the biggest benefit is a collective one; without the work of reviewers academia as we know it wouldn't work, and every researcher depends on other reviewers to check and improve their work.''}


% time vs learning experience and reputation (a careful report earns a good reputation with the associate editor)

% 2013 \it ``In the process of making fair decisions regarding paper acceptance one may develop foes in the same research triangle.  However referees venging personal grievances while reviweing papers should always be exposed and punished.''

\es \bs {\bf What are the costs and benefits of agreeing to review a paper?}

\it
``Costs are mainly the reviewer's time and effort. Benefits are to learn the latest development in a field, and perhaps when you are willing to review others' paper, others are also willing to take the time to review your paper.''


\es \bs As a researcher, one aims to read high-quality papers which have made, or will make, an impact. How can you estimate quality from (i) the journal reputation; (ii) the authors; (iii) internet sites such as Google Scholar and the Web of Knowledge ({http://webofknowledge.com/JCR}).

{\it ``(i) Journal reputation is important, since publishing papers on those journals would
be very strict and competitive, and the papers published on good journals are more
likely to be of good quality.''

``(ii) The authors is a good way to estimate, since if an author is well-estabilished,
he would have high criteria on himself and the work he published usually have good
quality.''

``(iii) Good scholar is a good reference, since it demonstrates how often one’s work is
cited by other, and this illustrates the significance of one’s work. But these quantified
measure could be not reliable sometimes.''
}

% 2014 \it  ``As for web resources, we should examine them carefully and critically.''


% ``(i) The papers published in a high-quality journal are more likely to be high-quality papers.
%
%(ii) The author who has better reputation are more likely to give high-quality papers.
%
%(iii) From internet sites, we can see the number of citation to judge.''


\es \bs Why do people try to assign ``credit'' between coauthors? How should one interpret the order of the authors? How is this affected by their reputations? 


{\it ``The assignation of credit may be a human tendency, but it also helps reward those people who worked the hardest on the paper. In statistics, the first author is generally given the lion’s share of the “credit” for the work–as the person who conceived and executed the majority of the research. Of course, author order may not accurately reflect the size of the contribution. Some authors may appear higher on the list due to seniority or for political reasons.''} 




\es
\bs
\it

``Assigning credit between coauthors is a way of generating proper peer recognition,
which is important in scientic career. The order of the authors is contingent on many
factors and since the authorship conventions may differ, it is not easy to interpret the
contribution of authors simply based on the order of the authors. But presumably,
we can infer that the author who comes first has made the greatest contribution.''

% 2014  ``... usually, if you read the name of a well-known author and other non known authors, you are going to assume that the well-known author has more credit in the research work.''

\es \bs 

\it
``Publications are very important in any researcher's career. So the credit between
coauthors becomes a big issue. Different disciplines and even different research groups
have their own tradition to assign orders of authors. So in order to interpret the
order, one must look into the specific situation. The reputations of authors usually
give readers a pre-notion that experienced researchers always make more contributions,
which might not always be the case.''

\es
\bs


When writing a manuscript, who should be included as an author? How and when is author order usually determined? 


{\it  Generally, a person should be listed as the author of a paper only if that person made a direct and substantial intellectual contribution to the design of the research, the interpretation of the data, or the drafting of the paper''. The more contribution one makes to the result, the higher his order should be.''}


\es
\bs

%Authors should make a substantial contribution.

A provisional author list should be determined as soon as possible. Why? 

\medskip

What is the consequence if you get involved in a paper when it is already partly finished?



\es \bs Collaboration: How much statistical advice should you give to a scientist before expecting the reward of coauthorship?

{\bf A}. {\it ``Verbal statistical advice, brief commentary, or pointing a scientist to the relevant literature need not merit coauthorship. However, if the statistician performs any data analysis and provides substantial statistical consultation then authorship may be merited and should at least be offered by the scientist.''}

\medskip

{\bf B}. {\it ``I would say it would need to be a relatively substantial contribution to the project. Ideally authorship should be agreed upon going into the project. Especially if statistical advice is used both for study design and analysis, I think authorship is appropriate. Or if the advice is really a major time investment for the statistician and the contribution to the project is large.''}

% 2014 \it ``A lot.''

\es

\bs
Collaboration: How much statistical advice should you give to a scientist before expecting the reward of coauthorship?

{\it ``If you have regular meetings (not one time advice), you should expect coauthorship.''}


\es \bs Generosity: What are the advantages and disadvantages of awarding coauthorship for relatively minor contributions? What are the advantages and disadvantages of refusing an offer of coauthorship if you feel your contribution is too small to justify it?

{\it ``For example, someone who has a high reputation in the field made a relatively minor contributions to the paper. You may still want to add him or her on the author lists so that people get more interested in your paper. In this case, however, those who made a major contributions may not get enough credit they deserve.''}

\medskip

{\it ``Awarding coauthorship for small contributions might encourage collaboration and openness among colleagues. It might also cheapen the publication process and re- duce accountability among the primary authors. Refusing an offer of coauthorship prevents you from being accountable for the results in a paper about which you know very little. Refusing an offer of coauthorship also discards an opportunity for public acknowledgment of your expertise.''}

\medskip

MY EXPERIENCE: THERE IS ALMOST ALWAYS A LOT OF WORK BETWEEN FINISHING A FIRST DRAFT AND THE FINAL PUBLISHED PAPER. ACCEPT CO-AUTHORSHIP IF YOU WANT TO FULLY PARTICIPATE IN THIS FUTURE WORK, NOT OTHERWISE. 



% 2014 \it``For the main authors of a paper, being overly generous on coauthorship can lessen the credit for the researchers with the most significant contributions. Advantages of awarding coauthorship lie in developing a reputation as a trusted, reliable researcher with whom others would want to collaborate with. Since all authors of a paper are considered responsible for the paper’s content, by refusing coauthorship, you no longer are responsible for a paper you may not have been able to truly validate.''

% 2013 (a) Awarding coauthorship for minor contribution: disadvantage- this might dilute the credit for other authors; advantage- give the collaborator credits for his/her work and might be good for future collaboration.

% 2013 (b) Refusing offer of coauthorship for minor contribution: disadvantage - you have fewer publication and your contribution is less recognizable; advantage- You might earn respect from other researcher. In addition, if the contribution is small, being a coauthor might earn less credit anyway but you have to take the responsibility for any errors or problems of the paper.

\es

\end{document}
